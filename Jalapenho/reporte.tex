\documentclass[letterpaper,11pt]{article}
\usepackage[spanish]{babel}
\usepackage[utf8]{inputenc}

\usepackage{rotating}
\usepackage{multirow}

\usepackage{lmodern}
\usepackage[T1]{fontenc}
\usepackage{textcomp}

\usepackage[
pdfauthor={Carlos Eduardo Caballero Burgoa},%
pdftitle={Reporte servidor},%
colorlinks,%
citecolor=black,%
filecolor=black,%
linkcolor=black,%
urlcolor=black
pdftex]{hyperref}

\title{Reporte Servidor}
\author{Carlos Eduardo Caballero Burgoa}

\begin{document}
\maketitle

\section{Introducción}
El presente reporte tiene por objetivo describir todas las actividades que se
realizaron alrededor del servidor prestado por Jala, por intermedio de la
carrera de Informática.

Inicialmente se discuten los entresijos para poner la maquina en funcionamiento,
se finaliza el documento describiendo los servicios que presto, y los
incovenientes actuales que posee.

\section{Descripción del equipo}
El equipo donado es una \emph{HP NetServer LPr} con las siguientes
caracteristicas:
\begin{itemize}
\item Intel Pentium III, con dos procesadores.
\item Memoria RAM de 1 GiB.
\item Dos discos duros SCSI de 20 GiB cada uno, con soporte para RAID.
\item Tarjeta de red Ethernet.
\end{itemize}

\section{Instalación}
Si bien podriamos mencionar que el servidor, es aún completamente funcional,
notamos un conjunto de barreras que dificultaron el trabajo, se hicieron dos
intentos de instalación, descritos a continuación:

\subsection{1º Intento (FreeBsd 8.2)}
Desde un comienzo se intento instalar FreeBSD 8.2 en la maquina, y en ese
momento notamos algunos problemas, sospechamos a un inicio del cd de instalación
o del lector, pero despues de probar con multiples copias, descubrimos que el
lector aún funciona, aunque puede no estar en sus mejores años.

Solventada la anterior intención, minimizando el uso del lector, con una
instalacion minimal, terminamos la instalación. Lo que nos llevo a nuestra
siguiente contingencia, el sistema se reiniciaba despues de un tiempo, sin
poderse determinar la causa exacta.

Partiendo de ese punto comenzamos a sospechar de algun componente de hardware,
principalmente de las RAM, cabe mencionar que la maquina posee 4 tarjetas tipo
DIM de 256 megabytes, para comprobar nuestra hipotesis, probamos las memorias
una por una.

Resultado de esto concluimos que 2 tarjetas funcionan correctamente, 1 de ellas
hizó el sistema extraordinariamente lento, y la 4ª reiniciaba el sistema despues
de un tiempo.

Tales experimentos nos llevaron a concluir que dos de ellas tenian sectores
defectuosos, que cuando el sistema operativo requiere esos puntos, ocasiona un
error de desbordamiento.

Una vez determinadas las memorias RAM, se reacomodaron de modo que se dio
prioridad a las que funcionaban correctamente, luego las memorias defectuosas,
despues de esto el sistema funcionaba correctamente a usos de tareas que no
requerian mucha carga en RAM; viendo la infame situacion se intento buscar una
solucion, para reutilizar las memorias defectuosas sin el riesgo de colgar el
sistema.

Se encontraron 2 metodos utiles, que funcionan con metodos muy parecidos
(basicamente implica señalar al kernel del sistema operativo los sectores
dañados de la RAM para que no los utilize), lamentablemente ambos metodos solo
funcionan con nucleos Linux, no encontrando nada similar para FreeBSD.

Y es asi que despues de tantas idas y venidas, iniciamos la instalacion de una
distribución Linux, considerando las apretadas restricciones de hardware en las
que nos encontramos, en la seleccion se priorizo la versatilidad ante otros
factores, es asi como llegamos a instalar Gentoo Linux (sin mencionar la
terrible obsesion del escritor por tal distribucion).

\subsection{2º Intento (Gentoo Linux)}
Una vez copiados los archivos e instalados todos los paquetes, se utilizó el
programa memtest86 para evaluar los sectores defectuosos de RAM siguiendo
patrones BADRAM, dando los siguientes resultados:

Despues de compilado un nucleo de linux personalizado para el hardware presente,
en otro ordenador, este se porto al servidor, colocando los parametros de badram
al nuevo kernel, terminando de esa manera la instalacion del nuevo servidor.

\subsection{Hardware presente}
Debido al alto grado de personalizacion que posee el nucleo compilado, se
detallan aqui los resultados del comando lspci, que nos muestra los componentes
de hardware y sus componentes de linux requeridos para funcionar.

\small
\begin{verbatim}
00:00.0 Host bridge:
    Intel Corporation 440BX/ZX/DX - 82443BX/ZX/DX Host bridge
        (AGP disabled) (rev 03)
    Flags: bus master, medium devsel, latency 64
    Memory at <unassigned> (32-bit, prefetchable)

00:04.0 ISA bridge:
    Intel Corporation 82371AB/EB/MB PIIX4 ISA (rev 02)
    Flags: bus master, medium devsel, latency 0

00:04.1 IDE interface:
    Intel Corporation 82371AB/EB/MB PIIX4 IDE (rev 01)
        (prog-if 80 [Master])
    Flags: bus master, medium devsel, latency 32
    [virtual] Memory at 000001f0 (32-bit, non-prefetchable) [size=8]
    [virtual] Memory at 000003f0 (type 3, non-prefetchable) [size=1]
    [virtual] Memory at 00000170 (32-bit, non-prefetchable) [size=8]
    [virtual] Memory at 00000370 (type 3, non-prefetchable) [size=1]
    I/O ports at 1050 [size=16]
    Kernel driver in use: ata_piix
    Kernel modules: ata_piix

00:04.2 USB controller:
    Intel Corporation 82371AB/EB/MB PIIX4 USB (rev 01)
        (prog-if 00 [UHCI])
    Flags: bus master, medium devsel, latency 32, IRQ 19
    I/O ports at 1060 [size=32]
    Kernel driver in use: uhci_hcd
    Kernel modules: uhci-hcd

00:04.3 Bridge:
    Intel Corporation 82371AB/EB/MB PIIX4 ACPI (rev 02)
    Flags: medium devsel, IRQ 9

00:07.0 PCI bridge:
    Digital Equipment Corporation DECchip 21152 (rev 03)
        (prog-if 00 [Normal decode])
    Flags: bus master, medium devsel, latency 57
    Bus: primary=00, secondary=01, subordinate=01, sec-latency=249
    I/O behind bridge: 00009000-00009fff
    Memory behind bridge: fa200000-fa2fffff
    Capabilities: [dc] Power Management version 1

00:09.0 Ethernet controller:
    Intel Corporation 82557/8/9/0/1 Ethernet Pro 100 (rev 08)
    Subsystem: Hewlett-Packard Company NetServer 10/100TX
    Flags: bus master, medium devsel, latency 66, IRQ 17
    Memory at fa100000 (32-bit, non-prefetchable) [size=4K]
    I/O ports at 1000 [size=64]
    Memory at fa000000 (32-bit, non-prefetchable) [size=1M]
    Capabilities: [dc] Power Management version 2
    Kernel driver in use: e100
    Kernel modules: e100

00:0d.0 VGA compatible controller:
    Cirrus Logic GD 5446 (rev 45) (prog-if 00 [VGA controller])
    Subsystem: Hewlett-Packard Company Device 0001
    Flags: medium devsel
    Memory at fc000000 (32-bit, prefetchable) [size=32M]
    Memory at fa101000 (32-bit, non-prefetchable) [size=4K]
    [virtual] Expansion ROM at 40000000 [disabled] [size=32K]

01:04.0 SCSI storage controller:
    LSI Logic / Symbios Logic 53c895 (rev 01)
    Flags: bus master, medium devsel, latency 247, IRQ 18
    I/O ports at 9000 [size=256]
    Memory at fa201000 (32-bit, non-prefetchable) [size=256]
    Memory at fa200000 (32-bit, non-prefetchable) [size=4K]
    Kernel driver in use: sym53c8xx
    Kernel modules: sym53c8xx
\end{verbatim}

\section{Servicios instalados}
El servidor posee instalados los siguientes demonios:

\begin{itemize}
\item EngineX como servidor web.
\item OpenSSH para acceso remoto.
\item PHP-FPM para acceso FastCGI de PHP.
\item Exim4 para servicios de correo electronico.
\end{itemize}

Se ha instalado Cacti para realizar la monitorizacion de los servicios de la
sociedad cientifica, ademas de ser utilizada como repositorio de copias de
respaldo de algunos servicios (bases de datos y logs de acceso principalmente)
en otros servidores.

\section{Inconvenientes encontrados}
Desde el inicio del trabajo con el servidor, se ha encontrado un problema que no
ha podido ser subsanado, es que al iniciar la maquina, esta presenta un error
en la administración de memoria, y debe de presionarse la tecla F1 para arrancar
normalmente el sistema, lo cual no fue un gran inconveniente hasta que se hizo
la puesta en producción del servidor.

Se noto que en la sala de servidores del FCyT es comun que se corte la luz, para
lo cual debe arrancarse nuevamente el sistema, y considerandose los
inconvenientes que posee la maquina, se lleva a tener la maquina con una
intermitencia inaceptable para los servicios a los que fue destinada
(monitorizacion).

\end{document}

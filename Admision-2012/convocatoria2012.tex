\documentclass[letter,12pt]{article}

\usepackage[T1]{fontenc}
\usepackage{lmodern}
\usepackage{textcomp}
\renewcommand*\familydefault{\sfdefault}

\usepackage[spanish]{babel}
\usepackage[utf8x]{inputenc}

\usepackage[pdftex]{graphicx}
\usepackage{pifont}
\usepackage[
pdfauthor={Carlos Eduardo Caballero Burgoa},%
pdftitle={Convocatoria 2013},%
colorlinks,%
citecolor=black,%
filecolor=black,%
linkcolor=black,%
%urlcolor=black
pdftex]{hyperref}

\usepackage{fancyhdr}
\usepackage{lastpage}
\pagestyle{fancy}

% Para la primera página
%\fancypagestyle{plain}{
%\fancyhead[l]{}
%\fancyhead[r]{}
%\fancyhead[c]{}
%\renewcommand{\headrulewidth}{0pt}
%\fancyfoot[l]{SCESI \\ Sociedad Científica de Estudiantes de Sistemas e Informática\\ \url {http://www.scesi.org}}
%\fancyfoot[c]{}
%\fancyfoot[r]{\thepage/\pageref{LastPage}}
%\renewcommand{\footrulewidth}{0.5pt}}

% Para el resto de páginas
%\lhead{}
%\chead{}
%\renewcommand{\headrulewidth}{0pt}
%\lfoot{SCESI \\ Sociedad Científica de Estudiantes de Sistemas e Informática \\ 
%\url {http://www.scesi.org}}
%\cfoot{}
%\rfoot{\thepage/\pageref{LastPage}}
%\renewcommand{\footrulewidth}{0.4pt}

\title{\bf ADMISIÓN POSTULANTES SCESI}
\date{}
\begin{document}
\maketitle
La Sociedad Cientifica de Estudiantes de Sistemas e Informática - SCESI, convoca a todos los estudiantes interesados que desean postularse para la convocatoria 2013.
Para esto deberán presentar los siguientes requisitos:

\begin{enumerate}
\item Presentar una fotocopia de Carnet de Identidad.
\item Ser estudiante, y estar actualmente inscrito en la cualquier carrera de la UMSS, presentando una fotocopia de matrícula de la gestión I/2013.
\item Presentar solicitud escrita dirigida a la comisión de admisión de la SCESI, en la cual se tiene que especificar los siguientes puntos:
\begin{itemize}
\item Mencionar si tiene una disponibilidad minima de 8 horas/semanales para la capacitación e investigación dentro de la sociedad científica.
\item Proponer un documento adjunto de un proyecto o un tema de investigación, ya sea individual o colectivo en el área de Sistemas-Informática en la que trabajará en el semestre.
\item Mencionar las areas que le interesan en el area de sistemas e informática.
\item Mencionar por que decidio postularse a la SCESI.
\end{itemize}

\item Tener disponibilidad de tiempo para asistir a todas las reuniones en el semestre 1/2013, las reuniones se realizan una vez a la semana los dias jueves de 14:15 a 17:15 pm. Según la disponibilidad de los integrantes.
\end{enumerate}
\newpage
\section*{Areas de interes}
Areas de elección para la evaluación de los postulantes a la SCESI
\begin{itemize}
\item Programación Web
\item Diseño web
\item Programación Funcional
\item Programación Movil
\item Marketing
\item Software Libre
\item GNU/Linux
\item Seguridad Informática
\item Base de Datos
\item Programación en Java
\item Servidores
\item Electrónica Básica
\end{itemize}

\section*{De la forma}
La documentación de presentarse foliada y engrampada.

\section*{De la fecha y lugar}
La documentación deberá ser presentada hasta horas 20:15 del día 28 de Marzo del 2013, en la Sociedad científica con cualquiera de sus integrantes.

\section*{Cronograma}
\begin{tabular}{|p{6.5cm}|l|}
\hline
Publicación de convocatoria & 08 de Marzo de 2013 \\
Presentación de documentos & hasta el 28 de Marzo de 2013 \\
Reunion informativa (Open season - auditorio MEMI - 14:30) & 15 de Marzo de 2013 \\
Entrevistas & del 01 al 05 de Abril de 2013 \\
Publicación de Resultados & 09 de Abril de 2013 \\
\hline
\end{tabular}

\section*{Valores deseables del postulante}
Durante el proceso de evaluación se tomara en cuenta los siguientes puntos:

\begin{itemize}
\item Tener conocimiento sobre la filosofía del software libre.
\item Apoyar a los proyectos.
\item Tener ganas de aprender e investigar.
\item Trabajo en equipo.
\item Compartir conocimiento.
\item Disponibilidad de tiempo.
\end{itemize}

\vspace{2.3cm}
\begin{minipage}{0.25\textwidth}
\begin{center}
-----------------------\\
{\bf Gonzalo Nina}\\
Presidente (scesi)\\
\end{center}
\end{minipage}
\begin{minipage}{0.47\textwidth}
\begin{center}
-----------------------\\
{\bf Carlos Caballero}\\
Comisión de Admisión
\end{center}
\end{minipage}
\begin{minipage}{0.28\textwidth}
\begin{center}
-----------------------\\
{\bf Ubaldino Zurita}\\
Director Académico\\
\end{center}
\end{minipage}

\end{document}          
